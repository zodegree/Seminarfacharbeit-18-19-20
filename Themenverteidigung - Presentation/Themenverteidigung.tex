\documentclass[11pt]{beamer}
\usepackage[utf8]{inputenc}
\usepackage[german]{babel}
\usepackage[T1]{fontenc}
\usepackage[usefilenames,% Important for XeLaTeX
  RMstyle=Light,
  SSstyle=Light,
  TTstyle=Light,
  DefaultFeatures={Ligatures=Common}]{plex-otf} %
\usepackage{amsmath}
\usepackage{amsfonts}
\usepackage{amssymb}
\usepackage{graphicx}
\usetheme{Dresden}
\author{Felix Beckmann, Leonhard Alkewitz, Max Lautenbach}
\title{Die Optimierung der Energiebilanz von simulierten Gebäuden mit Hilfe von evolutionären Algorithmen}
%\setbeamercovered{transparent} 
%\setbeamertemplate{navigation symbols}{} 
\logo{\includegraphics[scale=.1]{logo.png}} 
\institute{Spezialschulteil des Albert-Schweizer Gymnasium Erfurt} 
\date{\today} 
%\subject{} 
\begin{document}

\begin{frame}
\titlepage
\end{frame}

\begin{frame}
\begin{enumerate}
\item{Einführung in die Thematik der Optimierung und der Energiebilanz von Gebäuden}
\begin{enumerate}
\item{Verfahren zur evolutionären Optimierung}
\item{Ermittlung der Energiebilanz von Gebäuden}
\end{enumerate}
\item{Zielstellung der Seminarfacharbeit und Abgrenzung des Themas}
\item{Methodik zum Erreichen unserer Ziele und Vorstellung des Zeitplans}
\item{Motivation und Begründung zur Wahl dieses Themas}
\end{enumerate}
\end{frame}

\begin{frame}{Einführung in die Thematik der Optimierung und der Energiebilanz von Gebäuden}

\end{frame}

\end{document}