\documentclass[11pt]{beamer}
\usepackage[utf8]{inputenc}
\usepackage[german]{babel}
\usepackage[T1]{fontenc}
%\usepackage[usefilenames, RMstyle=Light, SSstyle=Light, TTstyle=Light, DefaultFeatures={Ligatures=Common}]{plex-otf} 
\usepackage{amsmath}
\usepackage{amsfonts}
\usepackage{amssymb}
\usepackage{graphicx}
\usepackage{multirow}
\usetheme{Dresden}
\author{Felix Beckmann, Leonhard Alkewitz, Max Lautenbach}
\title{Die Optimierung der Energiebilanz von simulierten Gebäuden mit Hilfe von evolutionären Algorithmen}
%\setbeamercovered{transparent} 
%\setbeamertemplate{navigation symbols}{} 
\logo{\includegraphics[scale=.05]{logo.png}} 
\institute{Spezialschulteil des Albert-Schweizer Gymnasium Erfurt} 
\date{\today} 
%\subject{} 
\begin{document}

\begin{frame}
\begin{center}
\pause \huge{2 376 000 000 000 000 $J$}  \\  
\pause \huge{641 000 000 $kWh$} \\ 
\pause \huge{81 000 000 $kg$ $SKE$ \footnote{SKE = Steinkohleeinheit}} \\ 
\hrulefill{}  \\
\pause \huge{2000-2500 $l$} \\ 
\pause \huge{10-15 $Badewannen$}
\end{center}
\end{frame}

\begin{frame}
\titlepage
\end{frame}

\begin{frame}
\begin{enumerate}
\item{Einführung in die Thematik der Optimierung und der Energiebilanz von Gebäuden}
\begin{enumerate}
\item{Verfahren zur evolutionären Optimierung}
\item{Ermittlung der Energiebilanz von Gebäuden}
\end{enumerate}
\item{Zielstellung der Seminarfacharbeit und Abgrenzung des Themas}
\item{Methodik zum Erreichen unserer Ziele und Vorstellung des Zeitplans}
\item{Motivation und Begründung zur Wahl dieses Themas}
\end{enumerate}
\end{frame}

\begin{frame}{Einführung in die Thematik der Optimierung und der Energiebilanz von Gebäuden}

\end{frame}

\begin{frame}{Methodik zum Erreichen unserer Ziele}
\begin{enumerate}
\item{Literaturrecherche für themenspezifisches Wissen}
\item{Wiederholung der objektorientierten Programmierung mit Hilfe von Java}
\item{Erstellung einer Simulation als Umgebung für Optimierung}
\item{Weiterentwicklung der Häuser mithilfe von computergestützten, evolutionären Optimierungsverfahren}
\item{Auswertung der Ergebnisse und Fehleranalyse}
\item{graphische Visualisierung der Ergebnisse}
\end{enumerate}
\end{frame}

\begin{frame}{Vorstellung des Zeitplans}
\begin{footnotesize}
\begin{tabular}{|p{2cm}||p{2.5cm}|p{2.5cm}|p{2.5cm}|} \hline
   Datum & Leonhard & Felix & Max \\ \hline \hline
   \multirow{2}*{Oktober 2018} & \multicolumn{3}{|p{7.5cm}|}{Literaturrecherche und Javakenntnisse auffrischen} \\ 
   & Bauphysik & 3D Graphik & evolutionäre Algorithmen \\ \hline
   November 2018 & Wissenstand über Architektur erweitern  & \multicolumn{2}{|p{5cm}|}{Ansätze der Simulation: Objekt Haus mit grundlegenden Eigenschaften}\\ \hline
   Dezember 2018 & Statikkenntnisse aneignen & Sachverstand über 3D Graphik erweitern & Rahmen für EA:\footnote{EA = evolutionärer Algorithmus} Einbinden in Simulation\\ \hline
   Januar 2019 & Physik des Energieverlustes & Modellierung von Hausmodulen\footnote{Haus wird aus mehreren Modulen aufgebaut} & Methoden der Vererbung von EA implementieren \\ \hline   
   Februar 2019 & \multicolumn{3}{|p{7.5cm}|}{Puffer} \\ \hline
\end{tabular}
\end{footnotesize}
\end{frame}

\begin{frame}{Vorstellung des Zeitplans}
\begin{footnotesize}
\begin{tabular}{|p{2cm}||p{2.5cm}|p{2.5cm}|p{2.5cm}|} \hline
   Datum & Leonhard & Felix & Max \\ \hline \hline
   März 2019 & wichtige Werte für Energiebilanz in Programm einbinden & Grundlage für Module in Programm einbinden & Verfassen Kapitel zu Optimierung \\ \hline
   April 2019 & Energiequellen herausarbeiten & \multicolumn{2}{|p{5cm}|}{Wissen über Architektur in Programm einbinden} \\ \hline
   Mai 2019 & Verfassen Kapitel zu Statik und Architektur & \multicolumn{2}{|p{5cm}|}{Sachverständnis zu thermalen Austausch von Haus in Simulation einbinden} \\ \hline
   Juni 2019 &  graphische Implementierung von Modulen & Verfassen der Kapitel zu thermalen Austausch & Methode zu Energiebilanzberechnung \\ \hline
   Juli 2019 & \multicolumn{3}{|p{7.5cm}|}{Puffer} \\ \hline
   August 2019 & Verfassen der Kapitel zu 3D Graphik & \multicolumn{2}{|p{5cm}|}{Beispieldurchlauf und Ergebnisdokumentation} \\ \hline
\end{tabular}
\end{footnotesize}
\end{frame}

\begin{frame}{Vorstellung des Zeitplans}
\begin{footnotesize}
\begin{tabular}{|p{2cm}||p{2.5cm}|p{2.5cm}|p{2.5cm}|} \hline
   Datum & Felix & Leonhard & Max \\ \hline \hline
   September 2019 & Visualisierung der Ergebnisse & \multicolumn{2}{|p{5cm}|}{Fehleranalyse und -berichtigung} \\ \hline
   Oktober 2019 & Visualisierung der Ergebnisse & \multicolumn{2}{|p{5cm}|}{Verfassen der Kapitel zu Simulation} \\ \hline
   November 2019 & Verfassen der Kapitel zu Simulation & Ausbesserung von graphischen Fehlern & Ausbessern von programmierten Fehlern \\ \hline
   Dezember 2019 & \multicolumn{3}{|p{7.5cm}|}{Korrekturlesen und Abgabe der Seminarfacharbeit} \\ \hline
   Datum & Felix & Leonhard & Max \\ \hline \hline
    Januar 2020 & \multicolumn{3}{|p{7.5cm}|}{Sicherstellung der Lauffähigkeit des Programms und graphische Vorbereitung auf Verteidigung} \\ \hline
     Februar 2020 & \multicolumn{3}{|p{7.5cm}|}{Erstellung der Präsentation und Beispielen} \\ \hline
      März 2020 & \multicolumn{3}{|p{7.5cm}|}{Vorbereitung des Kolloquiums} \\ \hline
\end{tabular}
\end{footnotesize}
\end{frame}
s
\end{document}
